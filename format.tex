% !TEX root = main.tex
\documentclass[12pt]{report}
\usepackage[utf8]{inputenc}

% Formatting
\usepackage[mmddyyyy]{datetime}
\usepackage{fullpage}
\usepackage{enumerate}
\usepackage[colorlinks=true,linkcolor=black,urlcolor=blue,bookmarksopen=true]{hyperref}
\usepackage{bookmark}

%\usepackage{subfiles}

\usepackage{xargs}
\usepackage{xparse}


% Math util stuff
% \usepackage{amsmath}
\usepackage{amsthm}
\usepackage[makeroom]{cancel}
\usepackage{siunitx}
\usepackage{mathtools} % loads asmmath

\sisetup{
    output-decimal-marker = {,},
    per-mode = symbol
}

\usepackage{pgfplots}
\pgfplotsset{compat=newest}
\pgfplotsset{
    axis lines=middle,
    axis line style={<->},
    every axis plot post/.append style={mark=none,samples=200,smooth},
    unit vector ratio=1 1 1
}

\everymath{\displaystyle}
\allowdisplaybreaks[1]

% Theorem styling
\theoremstyle{plain}
\newtheorem{theorem}{Theorem}[section]
\newtheorem{corollary}{Corollary}[theorem]
\newtheorem{lemma}[theorem]{Lemma}
\newtheorem{review}{Review}[section]

\theoremstyle{definition}
\newtheorem{definition}{Definition}[section]
\newtheorem{example}{Example}[theorem]

\theoremstyle{remark}
\newtheorem*{remark}{Remark}
\newtheorem*{note}{Note}

\DeclareMathOperator{\di}{d\!}

\newcommandx*\deriv[2][1=, 2=x, usedefault]{\dfrac{\di {#1}}{\di {#2}}}

\DeclareDocumentCommand \Eval { m m O{} } { \left.#1\right\rvert_{#2}^{#3} }

\renewcommand{\partname}{Chapter}
