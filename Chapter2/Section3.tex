% !TEX root = ../main.tex

\section{The Derivative of Polynomial Functions and $y = e^x$}
Recall that if $f(x) = x^2 + 2x + 1$, then $f'(x) = 2x + 2$. We could write this in different ways.
\begin{itemize}
    \item If $y = x^2 + 2x + 1$ then $y' = 2x + 2$.
    \item If $y = x^2 + 2x + 1$ then $\deriv[y] = 2x + 2$.
    \item If $y = x^2 + 2x + 1$ then $\deriv \left(x^2 + 2x + 1\right) = 2x + 2$.
\end{itemize}
\begin{remark}
    The last one, $\deriv$ is an instruction to take a derivative of what comes after it.
\end{remark}
\begin{theorem}[Derivative of a Constant]
    If $A$ is a constant and $f(x) = A$ then $f'(x) = 0$.
\end{theorem}
\begin{theorem}[Derivative of a line with a slope of 1]
    If $f(x) = x$ then $f'(x) = 1$.
\end{theorem}
\begin{review}
    Use the definition of a derivative to find $\deriv \left(x^2\right)$.
    \begin{gather*}
        \lim_{h \to 0}\left(\dfrac{\left(x + h\right)^2 - x^2}{h}\right) \\
        \lim_{h \to 0}\left(\dfrac{x^2 + h^2 + 2xh - x^2}{h}\right) \\
        \lim_{h \to 0}\left(\dfrac{h^2 + 2xh}{h}\right) \\
        \lim_{h \to 0}\left(h + 2x\right) \\
        0 + 2x \\
        2x
    \end{gather*}
\end{review}
\subsection{Basic Rules}
\subsubsection{Power Rule}
\begin{theorem}[Power Rule]
    If $n \geq 1$ is an integer, then
    \begin{align}
       \deriv \left(x^n\right) &= nx^{n-1} \\
       \deriv \left(Cx^n\right) &= (C \cdot n)x^{n-1} \qquad \text{If $C$ is a constant.}
    \end{align}
\end{theorem}
\begin{review}
    If $y = 5x^2$ find $y'$ using the definition of a derivative.
    \begin{align*}
        f'(x) &= \lim_{h \to 0} \left(\dfrac{5\left(x + h\right)^2 - 5x^2}{h}\right) \\
        f'(x) &= \lim_{h \to 0} \left(\dfrac{5\left(\left(x + h\right)^2 - x^2\right)}{h}\right) \\
        f'(x) &= \lim_{h \to 0} \left(\dfrac{5\left(x^2 + h^2 + 2xh - x^2\right)}{h}\right) \\
        f'(x) &= 5\lim_{h \to 0} \left(\dfrac{x^2 + h^2 + 2xh - x^2}{h}\right) \\
        f'(x) &= 5\lim_{h \to 0} \left(\dfrac{h^2 + 2xh}{h}\right) \\
        f'(x) &= 5\lim_{h \to 0} \left(h + 2x\right) \\
        f'(x) &= 5 \cdot 2x \\
        f'(x) &= 10x \\
    \end{align*}
\end{review}
\subsubsection{Constant Multiplication Rule}
\begin{theorem}[Constant Multiplication Rule]
    Suppose $F(x) = k \cdot f(x)$ for some real number $k$ if $f(x)$ is differentiable then $F(x)$ is also differentiable, and
    \begin{equation}
        F'(x) = k \cdot f'(x)
    \end{equation}
\end{theorem}
\begin{example}
    If $f(x) = \pi x^7$ find $f'(x)$.
    \begin{align*}
        f'(x) &= \pi \deriv \left(x^7\right) \\
        f'(x) &= \pi \cdot \left(7x^6\right) \\
        f'(x) &= 7\pi x^6
    \end{align*}
\end{example}
\subsubsection{Addition Rule}
\begin{theorem}[Addition Rule]
    If $F(x) = f(x) + g(x)$ and $f$ and $g$ are differentiable then $F(x)$ is also differentiable.
    \begin{equation}
        F'(x) = f'(x) + g'(x)
    \end{equation}
\end{theorem}
\begin{example}
    If $y=3x^5 - 7x^2 - \dfrac{1}{2}x + 5$ find $\deriv[y]$
    \begin{align*}
        y' &= 3 \cdot \deriv \left(x^5\right) - 7 \cdot \deriv \left(x^2\right) - \dfrac{1}{2} \deriv \left(x\right) + \deriv \left(5\right)\\
        y' &= 3 \cdot 5x^4 - 14x - \dfrac{1}{2} \cdot 1 + 0 \\
        y' &= 15x^4 - 14x - \dfrac{1}{2}
    \end{align*}
\end{example}
\subsection{Derivative of $f(x) = e^x$}
\begin{theorem}[Derivative of $f(x) = e^x$]
    \begin{equation}
        \deriv\left(e^x\right) = e^x
    \end{equation}
    If $f(x) = e^x$, then $f'(x) = e^x$.
\end{theorem}
