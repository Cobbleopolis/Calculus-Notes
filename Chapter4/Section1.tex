% !TEX root = ../main.tex

\section{Related Rates}
Problem asks for a rate of change of some quantity; use the rate of change of related quantities.
\begin{example}
    The volume of a right circular cone is given by
    \begin{equation*}
        V = \dfrac{1}{3}\pi r^2h
    \end{equation*}
    How does the volume change over time with respect to the change in height if $r$ is constant?
    \begin{align*}
        \deriv[][t]\left(V\right) &= \deriv[][t]\left(\dfrac{1}{3}\pi r^2 h\right) \\
        1 \cdot \deriv[V][t] &= \underbrace{\dfrac{1}{3}\pi r^2}_\text{constant} \cdot 1 \cdot \deriv[h][t]
    \end{align*}
    How is $\deriv[V][t]$ related to $\deriv[r][t]$ if h is constant?
    \begin{equation*}
        \deriv[V][t] = \underbrace{\dfrac{1}{3}\pi h}_\text{constant} \cdot 2r \cdot \deriv[r][t]
    \end{equation*}
    How is the volume changing over time if both the radius AND the height are changing?
    \begin{equation*}
        \deriv[V][t] = \underbrace{\dfrac{1}{3}\pi}_\text{constant}\left(r^2 \cdot \deriv[h][t] + 2r \cdot \deriv[r][t]h\right)
    \end{equation*}
\end{example}
\subsection{Practice}
If a snowball melts so that its surface area decreases at a rate of \SI{1}{\cm\squared\per\min}, find the rate at which the diameter decreases when the diameter is \SI{10}{\cm}. (Surface area of a sphere: $SA = 4\pi r^2$)\\
\includegraphics[scale=0.75]{Chapter4/Section1/Practice4_1_1.pdf}\\
\subsubsection{Solution}
Variables we know:
\begin{align*}
    d &= \text{ diameter} \\
    SA &= \text{ surface area} \\
    \deriv[SA][t] &= \SI{1}{\cm\squared\per\min}
\end{align*}
Variables we want:
\begin{equation*}
    \Eval{\deriv[d][t]}{d = \SI{10}{\cm}}
\end{equation*}
Equation:
\begin{align*}
    SA &= 4\pi r^2\\
    SA &= 4\pi \left(\dfrac{d}{2}\right)^2 \\
    SA &= \pi d^2
\end{align*}
Derive:
\begin{equation*}
    \deriv[SA][t] = \pi \cdot 2d \cdot \deriv[d][t]
\end{equation*}
Substitute:
\begin{gather*}
    \SI{1}{\cm\squared\per\min} = \pi \cdot 2\left(\SI{10}{\cm}\right) \cdot \deriv[d][t] \\
    \deriv[d][t] = \dfrac{\SI{-1}{\cm\squared\per\min}}{\SI{20\pi}{\cm}} \\
    \deriv[d][t] = \dfrac{-1}{20\pi}\text{ }\si{\cm\per\min}
\end{gather*}
The diameter is decreasing at $\dfrac{-1}{20\pi}\text{ }\si{\cm\per\min}$.\\
%Skipping Example 1 from slides.
\textbf{Important:} Be sure to analyze whether each peice of information given is a RATE or QUANTITY (distance, volume, etc).
\begin{itemize}
    \item \textbf{Rates} are ``d(blah)/dt''
    \item \textbf{Other quantities} are NOT
\end{itemize}
\begin{itemize}
    \item Remember to define your vairables!
    \item Include units.
    \item Answer the question! Don't just do the math.
\end{itemize}
\begin{example}
    A $\SI{5}{\ft}$ ladder, leaning against a wall, slips so that its base moves away from the wall at a rate of $\SI{2}{\ft\per\sec}$. How fast will the top of the ladder be moving down the wall when the base is $\SI{4}{\ft}$ from the wall?
    %TODO Create image.
    \begin{align*}
        x &= \text{base distance from the wall} \\
        y &= \text{height of contact with wall} \\
        \SI{5}{\ft} &= \text{Ladder length} \\
        \deriv[x][t] &= \SI{2}{\ft\per\sec}
    \end{align*}
    We want $\deriv[y][t]$ when $x = \SI{4}{\ft}$.
    \begin{note}
        We will be expecting a negative value.
    \end{note}
    Equation relating information:
    \begin{equation*}
        x^2 + y^2 = \left(\SI{5}{\ft}\right)^2
    \end{equation*}
    \begin{note}
        $\SI{5}{\ft}$ is a constant $x$ and $y$ are changing.
    \end{note}
    Take the derivetive with relation to $t$:
    \begin{equation*}
        2x \cdot \deriv[x][t] + 2y \cdot \deriv[y][t] = 0
    \end{equation*}
    \begin{note}
        It makes sense that the equation is equal to $0$ because $\SI{5}{\ft}$ is constant.
    \end{note}
    Substitute:
    \begin{equation*}
        2\left(\SI{4}{\ft}\right)\left(\SI{2}{\ft\per\sec}\right) + 2\left(y\right)\deriv[y][t] = 0
    \end{equation*}
    We have two unknowns, $y$ and $\deriv[y][t]$, but we can find $y$ when $x = \SI{4}{\ft}$:
    \begin{gather*}
        x^2 + y^2 = \left(\SI{5}{\ft}\right)^2 \\
        \left(\SI{4}{\ft}\right)^2 + y^2 = \SI{25}{\ft\squared} \\
        y^2 = \SI{9}{\ft\squared} \\
        y = \pm \SI{3}{\ft} \\
        y = \SI{3}{\ft}
    \end{gather*}
    Now back to the substitution:
    \begin{gather*}
        2\left(\SI{4}{\ft}\right)\left(\SI{2}{\ft\per\sec}\right) + 2\left(\SI{3}{\ft}\right)\deriv[y][t] = 0 \\
        \SI{16}{\ft\squared\per\sec} + \SI{6}{\ft} \cdot \deriv[y][t] = 0 \\
        \deriv[y][t]\SI{6}{\ft} = \SI{-16}{\ft\squared\per\sec} \\
        \deriv[y][t] = \dfrac{\SI{-16}{\ft\squared\per\sec}}{\SI{6}{\ft}} = -\dfrac{8}{3}\si{\ft\per\sec}
    \end{gather*}
    The ladder slides down at $\dfrac{8}{3}\si{\ft\per\sec}$.
\end{example}
%TODO finish 4.1
