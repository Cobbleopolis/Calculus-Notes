% !TEX root = ../main.tex

\section{Newton's Method}
How do we solve equations like $x^3 - 4x + 2 = 0$?
\begin{itemize}
    \item We could use a calculator\ldots But what about before calculators? And HOW do calculators solve it?
    \item We could factor it. But factoring polynomials of high degrees really isn't fun. Maybe it doesn't factor (i.e. $\sin^2(x) + 2x - 5$)
    \item There are a variety of \textit{numerical methods} that calculators and computers use to find roots of equations.
    \item One such method id Newton's Method.
\end{itemize}
\subsection{Intermediate Value Theorem}
\begin{theorem}[Intermediate Value Theorem]
    If $f$ is continouts on $[a, b]$ and $N$ is between $f(a)$ and $f(b)$, then there is some $c \in (a, b)$ such that $f(c) = N$.
    \begin{note}
        $c \in (a, b)$ means that the value $c$ is between $a$ and $b$
    \end{note}
    \includegraphics[scale=0.75]{Chapter3/Section4/ivtGraph.pdf}\\
\end{theorem}
\subsection{The idea behind Newtom's Method}
Suppose $f(x)$ is a polynomial.
\begin{itemize}
    \item Suppose we can apply the IVT to conclude that $f$ has a root between $a$ and $b$.
    \item We pick any value, $c_1$, between $a$ nd $b$, and use that as our first guess of what the zero might be.
    \item We probably won't be right because there are an infinite amount of numbers between $a$ and $b$.
    \item We can evaluate $f(c_1)$ using a calculator.
    \item We can find $f'(c_1)$ because we are awesome at derivatives\ldots right?
\end{itemize}
Because $c_1$ is an approximation and is probably wrong we need to be able to ge a more accruate estimate. To do that we take the tangent line of $f(c_1)$ and where it intersects the x-axis is $c_2$. Repeat the process to get a progressively more and more accurate estimate for the root of the function.
\begin{center}
    \includegraphics{Chapter3/Section4/newtonsMethod.pdf}
\end{center}
\subsection{A Recursive Formula}
If $c_1$ is our first estimate the we can get a better approximation $c_2$ (called the ``second approximation'') by finding
\begin{equation*}
    c_2 = c_1 - \dfrac{f(c_1)}{f'(c_1)}
\end{equation*}
The third approximation is
\begin{equation*}
    c_3 = c_2 - \dfrac{f(c_2)}{f'(c_2)}
\end{equation*}
We can keep going to get the $n$th approximation
\begin{equation}
    c_n = c_{n - 1} - \dfrac{f(c_{n - 1})}{f'(c_{n - 1})}
\end{equation}
\begin{note}
    In \textit{most} cases when computers (calculators, laptops, etc.) are asked to find the square root of a function they actually use Newton's Method to find the square root. For example, if we asked a computer to find $\sqrt{S}$ it would have the function $f(x) = x^2 - S$ and then applies Newton's Method to find the square root of $S$. In fact, \textit{most} processors have an instruction that is used to take the square root of a number and the most common implementation of this instruction uses Newton's Method and it is accurate up to 5 bits.
\end{note}
\begin{example}
    Verify that $f(x) = x^3 - 4x + 2$ has a root on the interval $(1, 2)$. Then use a first approximation of $c_1 = 1.5$ to find the third approximation of that root.
    \begin{gather*}
        f(1) = 1^3 - 4(1) + 2 = -1 \\
        f(2) = 2^3 - 4(2) + 2 = 2
    \end{gather*}
    The IVT says that this function \textit{must} have a root in $(1, 2)$.
    \begin{equation*}
        f'(x) = 3x^2 - 4
    \end{equation*}
    \begin{gather*}
        c_2 = c_1 - \dfrac{f(c_1)}{f'(c_2)} \\
        c_2 = 1.5 - \dfrac{f(1.5)}{f'(1.5)} \\
        c_2 = 1.5 - \dfrac{(1.5)^3 - 4(1.5) + 2}{3(1.5)^2 - 4} \approx 1.727
    \end{gather*}
    \begin{equation*}
        c_3 = 1.727 - \dfrac{(1.727)^3 - 4(1.727) + 2}{3(1.727)^2 - 4} \approx 1.678
    \end{equation*}
    The actual root is approximately 1.675, so we're correct up to two decimal places in just two iterations.
\end{example}
\begin{example}
    Verify that $f(x) = x^4 - x^3 + x - 2$ has a root on the interval $(1, 2)$. Then use a first approximation of $c_1 = 1.5$ to find the fourth approximation of that root.
    \begin{gather*}
        f(1) = 1^4 - 1^3 + 1 - 2 < 0 \\
        f(2) = 2^4 - 2^3 + 2 - 2 > 0
    \end{gather*}
    The IVT says that this function \textit{must} have a root in $(1, 2)$.
    \begin{equation*}
        f'(x) = 4x^3 - 3x^2 + 1
    \end{equation*}
    \begin{gather*}
        c_2 = c_1 - \dfrac{f(c_1)}{f'(c_1)} \\
        c_2 = 1.5 - \dfrac{(1.5)^4 - (1.5)^3 + 1.5 - 2}{4(1.5)^3 - 3(1.5)^2 + 1} \\
        c_2 \approx 1.3468
    \end{gather*}
    \begin{gather*}
        c_3 = c_2 - \dfrac{f(c_2)}{f'(c_2)} \\
        c_3 = 1.3468 - \dfrac{(1.3468)^4 - (1.3468)^3 + 1.3468 - 2}{4(1.3468)^3 - 3(1.3468)^2 + 1} \\
        c_3 \approx 1.3104
    \end{gather*}
    \begin{gather*}
        c_4 = c_3 - \dfrac{f(c_3)}{f'(c_3)} \\
        c_4 = 1.3104 - \dfrac{(1.3104)^4 - (1.3104)^3 + 1.3104 - 2}{4(1.3104)^3 - 3(1.3104)^2 + 1} \\
        c_4  \approx 1.3086
    \end{gather*}
    This is actually correct up to 4 decimal places!
\end{example}
\subsection{Good News and Bad News}
\subsubsection{Bad News}
Sometimes Newton's Method fails. For example, your $c_1$ might be at a point where there is a horizontal tangent and it won't intersect the x-axis
\subsubsection{Good News}
It doesn't fail very often, and when it works, it gets close to the right answer ``very quickly.''
