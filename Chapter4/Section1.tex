% !TEX root = ../main.tex

\section{Related Rates}
Problem asks for a rate of change of some quantity; use the rate of change of related quantities.
\begin{example}
    The volume of a right circular cone is given by
    \begin{equation*}
        V = \dfrac{1}{3}\pi r^2h
    \end{equation*}
    How does the volume change over time with respect to the change in height if $r$ is constant?
    \begin{align*}
        \deriv[][t]\left(V\right) &= \deriv[][t]\left(\dfrac{1}{3}\pi r^2 h\right) \\
        1 \cdot \deriv[V][t] &= \underbrace{\dfrac{1}{3}\pi r^2}_\text{constant} \cdot 1 \cdot \deriv[h][t]
    \end{align*}
    How is $\deriv[V][t]$ related to $\deriv[r][t]$ if h is constant?

\end{example}
\subsection{Practice}
If a snowball melts so that its surface area decreases at a rate of \SI{1}{\cm\squared\per\min}, find the rate at which the diameter decreases when the diameter is \SI{10}{\cm}. (Surface area of a sphere: $SA = 4\pi r^2$)\\
\includegraphics[scale=0.75]{Chapter4/Section1/Practice4_1_1.pdf}\\
\subsubsection{Solution}
Variables we know:
\begin{align*}
    d &= \text{ diameter} \\
    SA &= \text{ surface area} \\
    \deriv[SA][t] &= \SI{1}{\cm\squared\per\min}
\end{align*}
Variables we want:
\begin{equation*}
    \Eval{\deriv[d][t]}{d = \SI{10}{\cm}}
\end{equation*}
Equation:
\begin{align*}
    SA &= 4\pi r^2\\
    SA &= 4\pi \left(\dfrac{d}{2}\right)^2 \\
    SA &= \pi d^2
\end{align*}
Derive:
\begin{equation*}
    \deriv[SA][t] = \pi \cdot 2d \cdot \deriv[d][t]
\end{equation*}
Substitute:
\begin{gather*}
    \SI{1}{\cm\squared\per\min} = \pi \cdot 2\left(\SI{10}{\cm}\right) \cdot \deriv[d][t] \\
    \deriv[d][t] = \dfrac{\SI{-1}{\cm\squared\per\min}}{\SI{20\pi}{\cm}} \\
    \deriv[d][t] = \dfrac{-1}{20\pi}\text{ }\SI{}{\cm\per\min}
\end{gather*}
The diameter is decreasing at $\dfrac{-1}{20\pi}\text{ }\SI{}{\cm\per\min}$
%TODO finish 4.1
