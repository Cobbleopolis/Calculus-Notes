% !TEX root = ../main.tex

\section{Indterminate Forms and L'Hopital's Rule}
\subsection{Indeterminate Forms}
Let's look at the first two forms we're going care about here.\\
If
\begin{equation*}
    \lim_{x \to c}\left(f(x)\right) = \lim_{x \to c}\left(g(x)\right) = 0
\end{equation*}
then $\dfrac{f(x)}{g(x)}$ is an indeterminate form of the type $\dfrac{0}{0}$.\\
If
\begin{equation*}
    \lim_{x \to c}\left(f(x)\right) = \lim_{x \to c}\left(g(x)\right) = \infty
\end{equation*}
then $\dfrac{f(x)}{g(x)}$ is an indeterminate form of the type $\dfrac{\infty}{\infty}$.
\begin{note}
    $c = \infty$ is allowed
\end{note}
View these cases as we get \textit{no information} when we substitute to try and find the limit.
\begin{example}
    Evaluate: $\lim_{x \to 0}\left(\dfrac{\tan(x)}{x}\right)$ \\
    Remember:
    \begin{equation*}
        \lim_{x \to 0}\left(\dfrac{\sin(\theta)}{\theta}\right) = 1
    \end{equation*}
    \begin{gather*}
        \lim_{x \to 0}\left(\dfrac{\tan(x)}{x}\right) \subeq \dfrac{0}{0} \\
        \lim_{x \to 0}\left(\dfrac{\dfrac{\sin(x)}{\cos(x)}}{x}\right) \\
        \lim_{x \to 0}\left(\dfrac{\sin(x)}{x} \cdot \dfrac{1}{\cos(x)}\right) \\
        \lim_{x \to 0}\left(\dfrac{\sin(x)}{x}\right) \cdot \lim_{x \to 0}\left(\dfrac{1}{\cos(x)}\right)\\
        1 \cdot \dfrac{1}{1}\\
        1
    \end{gather*}
\end{example}
\subsection{L'Hopital's Rule}
\begin{theorem}[L'Hopital's Rule]
    Suppose $f$ and $g$ are differentiable near $c$, and $g'(x) \neq 0$ for all $x \neq c$ near $c$. Let $L$ be a real number (or $\pm \infty$), and suppose $\dfrac{f(x)}{g(x)}$ in an indeterminate form at $c$ of type $\dfrac{0}{0}$ or $\dfrac{\infty}{\infty}$. If
    \begin{equation}
        \lim_{x \to c}\left(\dfrac{f'(x)}{g'(x)}\right) = L
    \end{equation}
    then
    \begin{equation}
        \lim_{x \to c}\left(\dfrac{f(x)}{g(x)}\right) = L
    \end{equation}
\end{theorem}
The rule basically says \textbf{if you satart with a $\dfrac{0}{0}$ or $\dfrac{\infty}{\infty}$ indterminate form}, then taking the derivative of the top and the derivative of the bottom will not change the limit.\\
We are \textbf{NOT} taking the derivative of $\dfrac{f(x)}{g(x)}$!. Never take the derivative of a quotient without using the quotient rule.\\ %TODO reference quotient rule
You many need to apply the rule multiple times. You might get another indeterminate form when you take the derivative; so apply the rule again.\\
\textbf{YOU HAVE TO START WITH AN INDETERMINATE FORM!}
\begin{example}
    Evaluate: $\lim_{x \to 0}\left(\dfrac{\tan(x)}{x}\right)$ \\
    \begin{align*}
        \lim_{x \to 0}\left(\dfrac{\tan(x)}{x}\right) &\subeq \dfrac{0}{0} \\
        &\lheq \lim_{x \to 0}\left(\dfrac{\sec^2(x)}{1}\right) \\
        &\subeq \dfrac{\sec^2(0)}{1} \\
        &= \dfrac{1}{1} \\
        &= 1
    \end{align*}
    \begin{note}
        The ``L'H'' in $\lheq$ is required so we know that L'Hopital's Rule was used.
    \end{note}
    \begin{note}
        It is \textbf{NOT} true that $\dfrac{\tan(x)}{x} = \dfrac{\sec^2(x)}{1}$, so you really need the limit opperator to keep the staments equivelent.
    \end{note}
\end{example}
