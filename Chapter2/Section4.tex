% !TEX root = ../main.tex

\section{Product Rule, Quotient Rule, and Higher Order Derivatives}
\subsection{Basic Rules Continued}
\subsubsection{Product Rule}
\begin{theorem}[Product Rule]
    If $f$ and $g$ are differentiable then
    \begin{equation}
        \deriv\left(f(x) \cdot g(x)\right) = f(x) \cdot \deriv \left(g(x)\right) + \deriv \left(f(x)\right) \cdot g(x)
    \end{equation}
    Restated:
    \begin{equation}
        \left(f \cdot g\right)' = f \cdot g' + f' \cdot g
    \end{equation}
\end{theorem}
\begin{example}
    If $f(x) = e^x x^4$ find $f'(x)$.
    \begin{align*}
        f'(x) &= e^x \cdot 4x^3 + x^4 \cdot e^x \\
        f'(x) &= 4e^x x^3 + e^x x^4 \\
        f'(x) &= e^x x^3(4 + x) \\
    \end{align*}
\end{example}
\begin{example}
    If $y=4(x^2 - 7)$, find $y'$.
    \begin{align*}
        f'(x) &= 4x^2 - 28 \\
        f'(x) &= 8x
    \end{align*}
\end{example}
\subsubsection{Quotient Rule}
\begin{theorem}[Quotient Rule]
    If $f$ and $g$ are differentiable at $x$ and $g(x) \neq 0$ then $\dfrac{f}{g}$ is differentiable at $x$ and
    \begin{equation}
        \deriv\left(\dfrac{f(x)}{g(x)}\right) = \dfrac{g(x) \cdot \deriv\left(f(x)\right) - f(x) \deriv\left(g(x)\right)}{g(x)^2}
    \end{equation}
    Restated:
    \begin{equation}
        \left(\dfrac{f}{g}\right)' = \dfrac{g \cdot f' - f \cdot g'}{g^2}
    \end{equation}
\end{theorem}
\begin{remark}
    The order of the quotient rule can be remembered with the rhyme ``hi d lo lo d hi all over the square of what's below''.
\end{remark}
\begin{example}
    Find $\deriv\left(\dfrac{3x^3 - 5x}{5e^x + 2}\right)$
    \begin{align*}
        f' &= 9x^2-7 \\
        g' &= 5e^x
    \end{align*}
    \begin{equation*}
        \deriv\left(\dfrac{3x^3 - 5x}{5e^x+2}\right) = \dfrac{\left(5e^x + 2\right) \left(9x-7\right) - \left(3x^3 - 7x\right)\left(5e^x\right)}{\left(5e^x + 2\right)^2}
    \end{equation*}
\end{example}
\begin{remark}
    In some cases it is okay to not simplify the answer.
\end{remark}
\subsection{Revising the Power Rule}
\begin{align*}
    f(x)  &= x^n       & f(x)  &= Ax^n \\
    f'(x) &= nx^{x- 1} & f'(x) &= Anx^{n-1}
\end{align*}
Where $n$ is \underline{any} integer.
\begin{example}
    Find $f'(x)$ if $f(x) = \dfrac{1}{3x^4}$.
    \begin{align*}
        f(x) &= \dfrac{1}{3x^4} \\
        f(x) &= \dfrac{1}{3} x^{-4} \\
        f(x) &= -\dfrac{4}{3} x^{-5}
    \end{align*}
\end{example}
\subsection{Higher Order Derivatives}
\begin{definition}
    The derivative if $f'$ is the second derivative of $f$.

    \quad Notation: $f''(x)$

    \quad Read: ``$f$ double prime of $x$''

    \quad Can also consider third derivative $f'''$, fourth derivatives $f^{(4)}$, etc.
\end{definition}
\begin{example}
    If $f(x) = 5x^3$, find $f'$, $f''$, and $f'''$
    \begin{align*}
        f'(x) &= 15x^2 \\
        f''(x) &= 30x \\
        f'''(x) &= 30
    \end{align*}
\end{example}
\subsubsection{Why do we care?}
We know $f'(x)$ tells us the rate of change of $f$. What does $f''(x)$ tell us?
\begin{itemize}
    \item Rate of change of the rate of change\ldots
    \item In the context of $f = \text{position}$:
    \begin{itemize}
        \item $f'$ is velocity (how fast the position is changing)
        \item $f''$ is acceleration (how fast the velocity is changing)
    \end{itemize}
    \item In the context of $f = \text{number of unemployed people in the U.S.}$:
    \begin{itemize}
        \item $f'$ is how quickly unemployment is growing or shrinking
        \item Suppose we are in a recession where unemployment is increasing. As $f''$ decreases, it means that jobs are being more slowly.
    \end{itemize}
\end{itemize}
\begin{example}
    A rock thrown vertically from the surface of the moon at an initial velocity of 24 [m/s] reaches a height of $s = 24t - 0.8t^2$ meters in $t$ seconds
    \begin{enumerate}
        \item What is the velocity at time $t$? What is the acceleration?
        \item How long before the rock reaches its highest point?
        \item How high does the rock go?
        \item How long before the rock reaches half of it's maximum height?
        \item How long is the rock aloft?
        \item What is the rock's speed on impact?
    \end{enumerate}
    Answers:
    \begin{enumerate}
        \item \begin{align*}
            v &= s' = 24 - 16t [m/s] \\
            a &= s'' = -1.6 [m/s^2]
        \end{align*}
        \item \begin{align*}
            24 - 1.6t &= 0 \\
            24 &= 1.6t \\
            t &= 15 [s]
        \end{align*}
        \item \begin{gather*}
            24\left(15\right) - 0.8 \left(15\right)^2 \\
            180[m]
        \end{gather*}
        \item \begin{gather*}
            90 = 24\left(t\right) - 0.8\left(t\right)^2 \\
            0 = -0.8t^2 + 24t - 90 \\
            \dfrac{-24 \pm \sqrt{576 - 4 \cdot -0.8 \cdot -90}}{-1.6} \\
            \dfrac{-24 \pm \sqrt{288}}{-1.6} \\
            \dfrac{-24 \pm 12\sqrt{2}}{-1.6} \\
            t = \underline{4.3934}, 25.6066
        \end{gather*}
        \item $30 [s]$ (Two times the time to reach the peak (see \#2))
        \item $-24 [m/s]$ (Same as initial velocity but negative)
    \end{enumerate}
\end{example}
