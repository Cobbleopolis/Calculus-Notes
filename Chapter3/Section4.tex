% !TEX root = ../main.tex

\section{Newton's Method}
How do we solve equations like $x^3 - 4x + 2 = 0$?
\begin{itemize}
    \item We could use a calculator\ldots But what about before calculators? And HOW do calculators solve it?
    \item We could factor it. But factoring polynomials of high degrees really isn't fun. Maybe it doesn't factor (i.e. $\sin^2(x) + 2x - 5$)
    \item There are a variety of \textit{numerical methods} that calculators and computers use to find roots of equations.
    \item One such method id Newton's Method.
\end{itemize}
\subsection{Intermediate Value Theorem}
\begin{theorem}[Intermediate Value Theorem]
    If $f$ is continouts on $[a, b]$ and $N$ is between $f(a)$ and $f(b)$, then there is some $c \in (a, b)$ such that $f(c) = N$.
    \begin{note}
        $c \in (a, b)$ means that the value $c$ is between $a$ and $b$
    \end{note}
    \includegraphics[scale=0.75]{Chapter3/Section4/ivtGraph.pdf}\\
\end{theorem}
\subsection{The idea behind Newtom's Method}
Suppose $f(x)$ is a polynomial.
\begin{itemize}
    \item Suppose we can apply the IVT to conclude that $f$ has a root between $a$ and $b$.
    \item We pick any value, $c_1$, between $a$ nd $b$, and use that as our first guess of what the zero might be.
    \item We probably won't be right because there are an infinite amount of numbers between $a$ and $b$.
    \item We can evaluate $f(c_1)$ using a calculator.
    \item We can find $f'(c_1)$ because we are awesome at derivatives\ldots right?
\end{itemize}
Because $c_1$ is an approximation and is probably wrong we need to be able to ge a more accruate estimate. To do that we take the tangent line of $f(c_1)$ and where it intersects the x-axis is $c_2$. Repeat the process to get a progressively more and more accurate estimate for the root of the function.
\begin{center}
    \includegraphics{Chapter3/Section4/newtonsMethod.pdf}
\end{center}
\subsection{A Recursive Formula}
If $c_1$ is our first estimate the we can get a better approximation $c_2$ (called the ``second approximation'') by finding
\begin{equation*}
    c_2 = c_1 - \dfrac{f(c_1)}{f'(c_1)}
\end{equation*}
The third approximation is
\begin{equation*}
    c_3 = c_2 - \dfrac{f(c_2)}{f'(c_2)}
\end{equation*}
We can keep going to get the $n$th approximation
\begin{equation*}
    c_n = c_{n - 1} - \dfrac{f(c_{n - 1})}{f'(c_{n - 1})}
\end{equation*}
%TODO finish 3.4
