% !TEX root = ../main.tex

\section{Derivatives of Logarithmic Functions}
\subsection{Derivatives of Logarithmic Functions}
\begin{theorem}[Derivatives of Logarithmic Functions]
    If we recall that $\log_a(x) = y$, then $a^y = x$ as well as $\deriv\left(a^x\right) = a^x \ln(a)$ then the following must be true:
    \begin{equation}
        \deriv \left(\log_a(x)\right) = \dfrac{1}{x\ln(a)}
    \end{equation}
\end{theorem}
\begin{proof}
    \begin{align*}
        \log_a(x) &= y \\
        a^y &= x \\
        \deriv \left(a^y\right)    &= \deriv \left(x\right)\\
        a^y \ln(a) \cdot \deriv[y] &= 1 \\
        \deriv[y]                  &= \dfrac{1}{a^y \ln(a)} \\
        \intertext{This is fine but we don't want $y$ in our answer.}
        \deriv[y]                  &= \dfrac{1}{x \ln(a)} \\
        \intertext{This is valid because we know that $a^y = x$}
    \end{align*}
\end{proof}
What about $\deriv\left(\ln(x)\right)$?
\begin{align*}
    \deriv\left(\ln(x)\right) &= \deriv \left(\log_e(x)\right) \\
                              &= \dfrac{1}{x\ln(e)} \\
                              &= \dfrac{1}{x}
\end{align*}
\begin{note}
    The power rule will never give an exponent of $-1$, since $n = 0$ would mean that it's a constant.
\end{note}
\begin{example}
    Differentiate: $f(x) = \dfrac{x}{\ln(x)}$
    \begin{align*}
        f'(x) &= \dfrac{\left(\ln(x)\right)\deriv\left(x\right) - \left(x\right)\deriv\left(\ln(x)\right)}{\left(\ln(x)\right)^2} \\
              &= \dfrac{\left(\ln(x)\right)\left(1\right) - \left(x\right)\left(\dfrac{1}{x}\right)}{\left(\ln(x)\right)^2} \\
              &= \dfrac{\ln(x) - 1}{\left(\ln(x)\right)^2} \\
              &= \dfrac{\ln(x)}{\left(\ln(x)\right)^2} - \dfrac{1}{\left(\ln(x)\right)^2} \tag{Optional} \\
              &= \dfrac{1}{\ln(x)} - \dfrac{1}{\left(\ln(x)\right)^2} \tag{Optional}
    \end{align*}
\end{example}
\begin{example}
    Differentiate $y = \left|x\right|$
    \begin{equation*}
        y = \begin{cases}
            \ln(x) & x > 0 \\
            \ln(-x) & x < 0
        \end{cases}
    \end{equation*}
    \begin{note}
        Domain is all real number except 0
    \end{note}
    \begin{align*}
        x  &> 0:           & x &< 0: \\
        y  &= \ln(x)       & y &= \ln(-x) \\
        y' &= \dfrac{1}{x} & y' &= \dfrac{1}{-x} \cdot \deriv \left(-x\right) \\
           &               & y' &= \dfrac{1}{-x} \left(-1\right) \\
           &               &    &= \dfrac{1}{x}
    \end{align*}
    Conclusion: because $y'$ was the same in both cases then the following must be true:
    \begin{equation*}
        \deriv\left(\ln(\left|x\right|)\right) = \dfrac{1}{x}
    \end{equation*}
\end{example}
\subsection{Logarithmic Differentiation}
\subsubsection{Remember that question from the review?}
\begin{equation*}
    y = \left(\dfrac{x^4 \sin^2(x)}{\sqrt{1 - x^2}}\right)
\end{equation*}
Yeah that one. To derive this one normally would be a lot of chain, quotient, and power rules. However an esier way of taking the derivative by taking the natural log of both sides, then implicit derive of both sides and solve for $\deriv[y]$. This process is known as ``logarithmic differentiation''.
\begin{example}
    \begin{align*}
        y &= \left(\dfrac{x^4 \sin^2(x)}{\sqrt{1 - x^2}}\right) \\
        \ln(y) &= \ln\left(\dfrac{x^4 \sin^2(x)}{\left(1 - x^2\right)^{\frac{1}{2}}}\right) \\
        \ln(y) &= \ln\left(x^4 \left(\sin^2(x)\right)\right) - \ln\left(1 - x^2\right)^{\frac{1}{2}} \\
        \ln(y) &= \ln\left(x^4\right) + \ln\left(\sin(x)\right)^2 - \dfrac{1}{2}\ln\left(1 - x^2\right) \\
        \ln(y) &= 4\ln\left(x\right) + 2\ln\left(\sin(x)\right) - \dfrac{1}{2}\ln\left(1 - x^2\right) \\
        \dfrac{1}{y} \cdot \deriv[y] &= 4 \cdot \dfrac{1}{x} + 2 \dfrac{1}{\sin(x)} \cos(x) - \dfrac{1}{\cancel{2}} \cdot \dfrac{1}{1 - x^2} \cdot \cancel{2}x \\
        \deriv[y] &= y \left(\dfrac{4}{x} + 2\cot(x) - \dfrac{x}{1 - x^2}\right) \\
        \deriv[y] &= \dfrac{x^4 \sin^2(x)}{\sqrt{1 - x^2}} \left(\dfrac{4}{x} + 2\cot(x) - \dfrac{x}{1 - x^2}\right) \tag{Substitue the original function back in for $y$}
    \end{align*}
\end{example}
%TODO Finish section 3.3
